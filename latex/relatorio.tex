\documentclass[column,brazilian,12pt,a4paper,final]{article}
\usepackage[brazil]{babel}
%\usepackage{nonfloat}
\usepackage[utf8]{inputenc}
\usepackage{multicol}
\usepackage{hyperref}
\usepackage[pdftex]{color,graphicx}
\usepackage{geometry}
\usepackage{amsmath}
\usepackage{caption}

 \geometry{
 a4paper,
 total={170mm,257mm},
 left=20mm,
 top=20mm,
 }
 \usepackage{ragged2e}
\usepackage{fancyhdr}
\usepackage{caption}[=v1]
\usepackage{xcolor}
\usepackage{circuitikz}
\pagestyle{plain}
\fancyhead{}
\lhead{\includegraphics[width=1.5cm]{logoufrgs}}
\rhead{\includegraphics[width=0.8cm]{logoif}}
\fancyfoot{}
\fancyhead[C]{\footnotesize Física Experimental III $\bullet$ 2025/1}
\fancyfoot[RO]{\thepage}
\usepackage{float}
\usepackage{setspace}
\usepackage{fancyref}


\title{}
\author{Autores: \\ Lucas Assis Paulino da Silva - 590174 \\Lucas Bertazo de Deus Félix - 587064
 \\ Pedro Henrique Reis de Oliveira - 590908 \\ IF-UFRGS}
\date{Abril 2025}

\begin{document}
\maketitle
\thispagestyle{fancy}

\section*{Resumo}
\paragraph{}
[...]

\section{Introdução}
\paragraph{}
[...]

\section{Embasamento Teórico}
\paragraph{}
[...]

\section{Material Utilizado}
\begin{itemize}
    \item Fios, conectores, circuito
    \item Multímetro Minipa® ET-2075B (Precisão 0,001V e 0,01$\mu$F)
    \item Fonte Elétrica - IF UFRGS
    \item Capacitor (23,28$\mu$F)
    \item Cronômetro (Precisão 0,01s)
    \item Smartphone com câmera (Para gravação do vídeo)
\end{itemize}

\section{Procedimentos e Montagem}
\paragraph{}
[...]

\section{Dados Experimentais}
\paragraph{}
[...]

\section{Análise de Dados}
[...]
\section{Conclusão}
[...]

\begin{thebibliography}{99}

\bibitem{}
Processamento de dados e produção de gráficos:
\url{https://github.com/pedro-hro/Relatorio_3-ExperimentalIII}
\bibitem{}
RUTH W. CHABBAY. Matter and Interactions 4th Edition - Matter and Interactions, 4th Edition. WILEY, 2015.
\bibitem{}
NUSSENZVEIG, H. Moysés. {\em Curso de Física Básica - Mecânica}. 5ª ed., vol. 3. São Paulo: Edgard Blücher Ltda, 2013.
\bibitem{}
Schechner, S. J. (2015). The Art of Making Leyden Jars and Batteries according to Benjamin Franklin. ERittenhouse, 26. https://saraschechner.scholars.harvard.edu/publications/art-making-leyden-jars-and-batteries-according-benjamin-franklin


\end{thebibliography}

\end{document}